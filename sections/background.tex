\section{Background: Dynamic Symbolic Execution}
\label{sec:background}

Many state-of-the-art white-box test-generation tools rely on {\em dynamic
symbolic execution}~\cite{klee,dart-pldi05,sen-cacm}. This section provides a brief
overview of how these tools work, and discusses some of the shortcomings of
current techniques that \lancet aims to ameliorate.

\subsection{Dynamic Symbolic Execution Basics}

Dynamic symbolic execution couples the traditional strengths of symbolic
analysis---tracking constraints on variables to determine the feasibility of
various program behaviors---with the efficiency of regular concrete execution.
The inputs to a program are treated as symbolic variables.
% SB (11/12/13): This is better said as ``The inputs to a program are 
% treated as symbolic variables.''
As the program
executes, statements that involve symbolic variables are executed
symbolically, adding constraints on symbolic variables. When a
branch statement is reached ({\em e.g.} {\tt if (x < y) goto L}), {\em one} of
the paths is taken, and the appropriate constraint is added to the {\em path condition}, or set of constraints ({\em e.g.}, if the branch above is taken, the constraint $(x <
y)$ would be added to the path condition). As new constraints are added to the
path condition an SMT solver (Satisfiability Modulo Theory) is invoked to ensure that
the path condition is still satisfiable; unsatisfiable constraints imply that
the particular path that execution has taken is infeasible---no program
execution could follow that path. Once a path through the program is found,
the SMT solver is used to produce a {\em concrete} set of values for all
the symbolic variables. These concrete values constitute an input. Note that
unlike traditional symbolic execution, dynamic symbolic analysis can always
fall back to concrete values for variables; if execution encounters a
statement that cannot be analyzed using the underlying SMT solver, calling an external function for example, the
variables involved can be concretized and the statement can be executed
normally. What it gives up is completeness of the code coverage as a result
of the concretization. 

\subsection{Path Exploration Heuristics}

One of the key decisions in dynamic symbolic execution tool is how to choose
the path through a program an execution should explore. In other words, how to
choose directions for the branches encountered in execution. For example, a
tool such as KLEE~\cite{klee} executes multiple paths concurrently, in an attempt to
generate a series of inputs to exercise different paths of the program.
When encountering the branch statement described above, KLEE can fork the
execution to two clones that follow the two branches respectively, choose one
execution clone to explore first and save the other for later. The resultant
paths would include the constraint $(x < y)$ in the clone that follows the true
branch, and $\neg(x < y)$ in the one that follows the false
branch.


There are numerous heuristics that can be used to choose which path
to explore first at each branch in dynamic symbolic execution; the choice of the
correct heuristic depends on the goals of the particular tool being developed.
We abstract away the goal of a dynamic symbolic execution tool by describing
it in terms of a {\em meta-constraint}. A meta-constraint is a higher level
constraint that the path condition describing a particular execution
attempts to satisfy. For example, in the case of generating high-coverage
tests, the meta-constraint for a tool may be to produce a path that exercises
program statements not seen by previously-generated inputs, in which case a
path-exploration heuristic might prioritize flipping branches to generate a
never-before-seen set of constraints. In the case of generating stress tests,
as in \lancet, the meta-constraint would be to generate a path condition
that exercises a particular loop body a certain, user-defined number of times, in which case
the path-exploration heuristic might prioritize taking branches that cause the
loop to execute again, till the user-defined limit is reached 
(Section~\ref{sec:loop-centric}). Note that just
as a path condition may not be tight---there can be many possible
concrete inputs that follow a particular path through the execution---a
meta-constraint need not be tight: multiple path conditions may all
satisfy a given meta-constraint.

\subsection{Dynamic Symbolic Execution Overhead}

The primary drawback of dynamic symbolic execution is its dependence on an
underlying SMT solver to manage the path condition that an execution generates.
At every branch, the SMT solver must be invoked to determine whether a
particular choice for a branch is feasible or infeasible. Though the
constraint solver is also invoked to concretize input variables when
necessary, or to generate the final concrete input for a particular run, the
majority of queries to the solver arise from these feasibility
checks~\cite{sen-cacm}. Though there are many techniques to reduce the expense
of queries to the constraint solver, such as reducing the query size by
removing irrelevant constraints and reusing the results of prior queries
whenever possible~\cite{klee}, the fundamental expense of invoking the
constraint solver at each branch in an execution remains. This overhead can
lead to an order-of-magnitude slowdown in execution~\cite{klee}.

A second overhead of dynamic symbolic execution is the path-explosion problem.
The path exploration heuristics of a particular tool may prioritize certain
choices for branches in an attempt to satisfy the tool's meta-constraint.
However, the exploration heuristics may be wrong: whenever a branch is
encountered, if the heuristic makes the wrong choice, the meta-constraint may
not be satisfiable, and the tool must return to the branch to make a different
choice.

These two overheads make generating large-scale, long-running inputs using
dynamic symbolic execution difficult. First, as the path being explored gets
longer, more branches are encountered, resulting in more invocations to the
SMT solver. Second, long-running inputs execute more branches, creating more
opportunities for the path-exploration heuristic to ``guess wrong,'' leading
to unsatisfiable meta-constraints and ultimately leading to path explosion.

\subsection{WISE}

WISE is perhaps the most closely-related dynamic symbolic execution technique
to \lancet. WISE attempts to generate worst-case inputs for programs ({\em
e.g.}, a worst-case input for a binary-search-tree construction program would
be a sorted list of integers, generating an unbalanced tree)~\cite{burnim-icse09}. In
meta-constraint terms, WISE attempts to generate a path condition
that produces worst-case inputs of a reasonably large size.

While it is possible to exhaustively search through all possible path conditions for a certain input size to find the worst-case inputs, it is clear that this approach will lead
to path explosion when applied to larger inputs. To avoid the path explosion
problem, WISE uses the following strategy. It exhaustively searches the
input space for small input sizes to find worst-case behaviors. Then, by
performing pattern matching on the worst-case path conditions generated
for different input sizes, WISE learns what choices worst-case path conditions typically make for branches in the program ({\em e.g.}, always following the
left child to add a new element in a BST). This pattern is then integrated
into a path-exploration heuristic.

At large scales, when WISE encounters a branch, it looks through the
patterns that it identified at small scales to choose the direction for the
branch. In essence, WISE has a notion of what a worst-case path condition
``looks like,'' and chooses directions for branches to make the path condition
for a larger input match that worst-case pattern. By using this new
path-exploration heuristic, WISE is able to find worst-case inputs for larger
scales without exploring every possible path. In many cases, WISE need only
explore a single path through the program to find a worst-case input!

WISE addresses the overheads of dynamic symbolic execution by tackling the
second drawback described above: it controls path explosion by learning the
pattern of path constraints from smaller runs where path explosion is less of
an issue. Nevertheless, WISE still sits on top of dynamic symbolic execution,
and must query the SMT solver at every branch; it does not tackle the first
source of overhead. As a result, even though WISE has much better scalability
than na\"ive symbolic execution, it still cannot generate particularly large
inputs. Generating an input that runs a loop one million times still requires
performing symbolic execution on a run that visits the loop test condition one
million times.


Section~\ref{sec:inference-mode} explains how \lancet tackles this
problem. In particular, \lancet can generate large-scale inputs for a program
{\em without ever performing symbolic execution at large scales}.
